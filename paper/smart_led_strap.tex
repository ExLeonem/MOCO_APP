\documentclass[conference]{IEEEtran}
\IEEEoverridecommandlockouts
% The preceding line is only needed to identify funding in the first footnote. If that is unneeded, please comment it out.
\usepackage{cite}
\usepackage{amsmath,amssymb,amsfonts}
\usepackage{algorithmic}
\usepackage{graphicx}
\usepackage{float}
\usepackage{textcomp}
\usepackage{xcolor}
\usepackage[utf8]{inputenc}
\def\BibTeX{{\rm B\kern-.05em{\sc i\kern-.025em b}\kern-.08em
   T\kern-.1667em\lower.7ex\hbox{E}\kern-.125emX}}
\begin{document}


\title{Wake alarm - Using a react-native client in combination with a rust backend.}

\author{\IEEEauthorblockN{1\textsuperscript{st} Maksim Sandybekov}
\IEEEauthorblockA{\textit{computer science - autonomous systems)} \\
\textit{HTWG Konstanz}\\
Konstanz, Germany \\
maksim.sandybekov@live.de}
\and
\IEEEauthorblockN{2\textsuperscript{nd} Benjamin Bäumler}
\IEEEauthorblockA{\textit{computer science - autonomous systems} \\
\textit{HTWG Konstanz}\\
Konstanz, Germany \\
be391bae@htwg-konstanz.de}
}

\maketitle

\begin{abstract}
With a dynamic market the area of smart lights is steady evolving and new innovations appear regularly. Comparing the number of
different devices, it is fairly easy to implement such a device accompanied by an application using today's technologies
and public available hardware. An introductory describing the effects of light on humans supports the definition of requirements
and functionality. Followed by an architectural overview that depicts the components and their interaction with each other, which
can be used to define an analogous architecture within another infrastructure of technologies. As we proceed we define the
components and their functionality themselves explaining our thought process and the decisions we made along the way.
Finally leading to an outlook of additional features which could and should be added.

\end{abstract}

\begin{IEEEkeywords}
rust, react-native, redux, redux-saga, light, alarm, smart-light, wake light
\end{IEEEkeywords}

% Motivation, why are we doing the likes
\section{Introduction}
The quality and duration of sleep affects the health and well being of individuals.
Additionally sleep plays a major role in consolidation of memory \cite{Rauchs2005} therefore it is essential for learning processes.
Looking at studies on sleep deprivation and disorders it becomes clear that a poor sleep can cause an decrease in both
mental and physical performance. \cite{Mirghani2015a, Antunes2017a} Leading up to various physical and mental
diseases for example type-2 diabetes, anxiety and increased depression. Especially students represent a group that
is likely to suffer severe sleep deprivation. In an conducted study, 46\% of 546 students rated their sleep as fairly bad up to
very bad. Furthermore 33\% of participants reported $\leq$ 7 h of sleep on study days with an average of 6.55 h. \cite{Norbury2019a}
The research surrounding the impact of light on the human circadian clock thereby revealed an relationship between illuminance and
alertness in human beings. \cite{DuffyJeanne2009a} In addition further research uncovered a link between exposure to more intense
light and the feeling of vitality during daytime and everyday situations. \cite{Smolders2014a}
These insights suggests a system that utilizes the effects of light on the human body to improve alertness and mental as well as
physical performance during the day. Using these mechanisms within the context of circadian stimulation and sleep, different
fields of application become obvious.
While technological development proceeds there are already attempts utilizing current innovations to harness prior introduced
positive effects. One such attempt are wake lights that simulate the sunrise. A study investigating effects on the human body
concludes that simulation of the dawn significantly improves performance on attention- and motor-based tasks/skills during the day.
\cite{Gabel2015a}
This paper proposes an application implementing functionality to utilize previously introduced advances in research
concerning the effects of light on the human body. The application enables a user to obtain control over a smart lamp
to regulate the brightness and color. An additional system for scheduling of illumination facilitates the simulation of sunrises.

\section{State of the art}
Currently there is a broad range of available commercial products depicted as smart lamps. Most of them representing platforms.
Besides providing the lamp itself they often offer REST based API's, mobile applications to control the device and integratable
third party software like IFTTT. These platforms differ in ways in which they offer their functionality and are accessible.
The main technologies used to realize connectivity in these devices are bluetooth, Wi-Fi, sub-GHz, IEEE802.15.4 and VLC. 
Smart lighting systems focused on short range connectivity however primarily use 6loWPAN or ZigBee Light Link. The core
functionality such devices offer are the remote control of the light itself, color and brightness. Additionally there remain advanced
algorithms like monitoring of light spectrum, daylight levels, user occupcation to decide further actions and color reproduction. \cite{higuera2018}
The latter feature providing means to influence humans circadian system and ultimately lifting concentration and mood of users.
Other types of devices focus more on niche use-cases like \cite{nur2018,philips19} acting as a wake/sleep light. As smart lighting
systems gain in popularity the area of security becomes more important. Looking at the current state of security in such devices is
worrisome. In literature on the security of smart lighting, attacks are known and successful. \cite{Morgner2017,Eichelberger2015}
To assume that the obtainable data has no private character is incorrect. There are already approaches to exfiltrate
users private data through a gateway that could be opened because of smart lights lacking the required counter measurements. \cite{Maiti2018}

% Approach
\section{Proposed approach}
To enable users to make full use of a lightning system, there is a need for the right control mechanisms to be in place.
Basic operations a user may take are activation/deactivation, change of color and brightness of the lamp. The context of use can
be broadened by adding the possibility to schedule the activation of the light. This operation enables the use as a visual alarm
clock or an automated lighting system.
We propose a distribution into client and server, already seen in different application. The client resembles a mobile application
offering functionality. On the contrary the server represents the device itself communicating with the client to offer an interface for
interaction with the light.
The communication between both instances happens through a wireless-network. 
%To results?
As we are only interested to visualize the possibility
to develop a universally usable lighting system, a security layer will not be defined but should be in a production build.


\subsection{Architecture}
On an architectural level the application firstly is divide into client and server as depicted in figure \ref{topLevelArch}.
The communication between these two instances happens via http. The server thereby provides an api for data insertion and
querying.

\begin{figure}[H]
   \centering
   \includegraphics[width=0.4\textwidth]{top_level_architecture}
   \caption{Top level architecture}
   \label{topLevelArch}
\end{figure}

Looking at the clients architecture depicted in figure \ref{clientArch}, there is a clear distinction between upper and below
part. The upper block of components may be compared to the view layer in traditional web application architectures. In contrast
the below part can be thought of as an mixture of controllers and views. Realizing the functionality
for persistence of data, application state as well as querying, insertion of schedule and led data at the server side.

\begin{figure}
   \centering
   \includegraphics[width=0.4\textwidth]{client_architecture}
   \caption{Client architecture}
   \label{clientArch}
\end{figure}

The server is split into two main components depicted in figure \ref{serverArch}, a web-framework is used to provide api endpoints that can be accessed through http by
the client. It uses multiple worker threads to handle requests in case we need to handle many requests in a short duration.
The controller thread handles the logic of the application it checks if a schedule should run and controls the light source.
Both components communicate never directly with another and use instead the \texttt{LedCache} object or the Schedule Database.

\begin{figure}[H]
   \centering
   \includegraphics[width=0.3\textwidth]{server_architecture}
   \caption{Server architecture}
   \label{serverArch}
\end{figure}

\subsection{Client}
% UI Components in correlation with the basic operations
% realization on technical level
The client is purely written in react-native/javascript. This technology enables cross-plattform development of mobile applications
while offering rather quick development cycles. Apart from programmatically differences this library requires another mindset than
native development. Compared to other technologies the user-interface consists of so called components each of which is made up of
native elements. The developed components may be reused and composed to create new components. In addition it is possible to pass
properties to a component. These are comparable with xml attributes, inside the component itself they can be accessed as an object.
In our case we divided the visuals in atomic components representing for example input fields and buttons. Subsequently building
heavier components compounded out of this smaller parts. At the top most level, we simply have screen components that consist of
other components themselves. Each of these screens providing a different functionality to the user. Following are the main screens
listed.

\vspace{5pt}
\begin{itemize}
   \item \textbf{DeviceSetupScreen:} setting up a new device
   \item \textbf{LightScreen:} activate/deactivate light and set brightness
   \item \textbf{ColorSelectionScreen:} change color of the led
   \item \textbf{scheduleScreen:} schedule led activation
\end{itemize}
\vspace{12pt}

% as there can be identified different parts of the application
% divide into part (presentation layer (components), store (similar to model), )

\subsubsection{Navigation}
As the application consists of multiple screens and react-native doesn't provide a native possibility to navigate between screens an
additional library is needed. At present there are mainly two libraries available react-native-navigation and react-navigation. These
libraries mainly differ in the interface provided for navigation and their performance. The selection in our case was of agnostic nature,
preferring the one with a shallower learning curve.

\subsubsection{State-management}
While a component is able to maintain state, looking at compound components it may appear that the same state needs to be managed at
different spots. A far better practice for this occurrence involves the usage of an additional library designed especially for
state-management. As this question often arises in javascript applications, different attempts have been made to solve this issue.
Many of which is the redux library, which we selected because it is well documented and easily integratable with react-native.
Representing a predictable-state container the redux library centralizes state in the so called store.
For a component to access the state it first must connect to the store. Afterwards a mapping must be defined of state variables onto the
property object of that specific component. To change state variables actions needed to be defined which are getting dispatched by redux to
reducers. While actions represent simple javascript functions returning an object with a minimum of a type field and additional payload,
the reducers are listening for different action types and using the payload to transfer the state.

\subsubsection{Communication/side effects}
For the communication with the server we limited ourself to the http protocol. As there is no network discovery implemented the user needs
to know the IP address of the server he wants to connect to. To ease the task at hand and keep a clear separation from the presentation
layer we use the redux-saga library. Like the name suggest this library builds upon redux. This library mainly listens to dispatched
actions and triggers generators which execute side effects such as http calls to the server or database writes and reads. These in
return can dispatch actions themselves.
The applications uses this functionality to acquire the current schedules from the server, as well as the current state of the device
for example the led brightness and color of the led. These values are stored in the redux-store of the device.

\subsection{Server}
The main tasks of the server is to communicate with clients and to actually drive an led strip or any other kind of light source.
This is realised through two main components a web framework to serve a REST-api that is used to communicate with clients and a
controller running in a separate thread, containing all the logic and managing the light source.

\subsubsection{Concurrency}
The execution flow of the controller and the web framework should be independent. We achive that by utilizing threads. Using Rust
allows us to confidently write multithreaded applications\cite{rust:con}, by preventing data races in the first place and with its
provided concurrency primitives. Message passing is used to communicate between the web framework and the controller. The controller
loop checks each time at the beginning if it received a message from the web framework and processes it. Currently the
controller handles five different messages:

\begin{itemize}
   \item GetData:
   The web framework needs the current led data to response to a request.
   \item UpdateOn, UpdateColor and UpdateBrightness:
   The controller should use the new provided value and use it. In case a schedule is running it will be stopped.
   \item DatabaseChanged:
   The web framework modified the database containing all schedules, the controller should check if the modification is relevant
   for it.
\end{itemize}

The web framework only expects two messages:
\begin{itemize}
   \item DataChanged:
   Some data changed since the last time the web framework updated its copy and should update it again.
   \item DataDump:
   Contains all data relevant to led status. The web framework needs to actively request it with \texttt{GetData}. 
   The request needs to be done because the controller has no knowledge of the moment the data is needed. On the contrary
   sending the data each time something changes would be an option, although this may overflow the message queue because
   it only gets checked when an api request is received. 
\end{itemize}

The web framework doesn't communicate with the controller directly and uses a \texttt{LedCache} instead. Through it the web
framework can access the current values of the light source or change them. The \texttt{LedCache} makes sure that it only communicates
with the controller thread when necessary.


\subsubsection{Light Sources}
The server can support different kinds of light sources, as long as it implements the \texttt{LedControls} trait. However only one
implementation of the trait can be used at the same time. Still it is possible that a specific implementation controls multiple
hardware lights. At the current time there are only two implementations that can be used as light sources:

\begin{itemize}
   \item LedStrip
   Controls a 4-pin led strip with a 12v pin and one pin for each color(red, green,blue). This cannot be driven directly by the
   Raspberry Pi and therefore we need to use a extra circuit board\cite{rpiled} for it. The circuit board allows us to control each
   color separately by driving the gate pin of a MOSFETs respectively. We use the pigpio Daemon\cite{pigpiod} for this functionality.
   As we need 3 PWM pins and the currently available gpio libraries for Rust only over up to 2 PWM pins.
  
   \begin{figure}[H]
       \centering
       \includegraphics[width=0.4\textwidth]{board}
       \caption{Raspberry Pi and custom controller board for 4-pin led strips}
   \end{figure}

   \item MocLedStrip
   This is used only for testing. It allows us to verify the logic of the led controller without driving any GPIO's of the
   Raspberry Pi and also allows us to run the server on amd64 architecture for testing purposes.
\end{itemize}


\section{Results}
The current implementation covers the basic operations, allowing the user to control the light source and set a schedule
for the visual alarm.
Because the client is written in react-native the portability on iOS and Android devices is ensured. Leading to the conclusion
that a broader range of potential users can be targeted as with conventional mobile development for a single platform.
The server executable is very lightweight. It has less than 5 MB memory usage and sits at 0\% cpu consumption when no web
requests are handled. Which is important because it will run non stop and ideally we want to use as less energy as possible while
doing nothing. The largest performance hog is the pigpiod daemon, which may be minimized by only running it when needed or by moving
the gpio functionality directly into the server application.


\section{Conclusion}
As we have shown how a smart light can be constructed using innovative technologies and easy accessible hardware, it is fairly
easy to construct such a device. Which in fact could be used to explore further possibilities. Exemplarily a device could be
implemented with a different set of functionality to condition the human circadian system exploring the possible range of effects.
Ultimately contributing to innovations in the field of smart lightning.


\section{Further work}
As the current implementation covers only the basic operations, the user pretty much is lost in setting the ideal parameters
for the light.
To gain the effects described in the introduction, guidance of the user could be added to facilitate
optimal use of the light. With the provided information on light configurations and their effects on the human body
the user could make better decisions depending on the desired effect. Another possibility would be to offer preconfigured
light settings for specific effects like lifted mood oder alertness.
An additional feature could include the automated discovery of the smart light device through the client, providing a
better usability.
Furthermore the functionality of the schedule system could be extended to integrate the local alarm clock as well as the calendar
of the mobile device. The resulting information could be used to prematurely react to events and trigger the light or enable the user
to add a schedule based on this information.
Finally, appropriate security measurements should be implemented to secure private data of the user and decrease vulnerability of the device.

\bibliographystyle{amsplain}
\bibliography{moco}

\end{document}