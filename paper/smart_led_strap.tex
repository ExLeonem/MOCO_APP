\documentclass[conference]{IEEEtran}
\IEEEoverridecommandlockouts
% The preceding line is only needed to identify funding in the first footnote. If that is unneeded, please comment it out.
\usepackage{cite}
\usepackage{amsmath,amssymb,amsfonts}
\usepackage{algorithmic}
\usepackage{graphicx}
\usepackage{textcomp}
\usepackage{xcolor}
\usepackage[utf8]{inputenc}
\def\BibTeX{{\rm B\kern-.05em{\sc i\kern-.025em b}\kern-.08em
    T\kern-.1667em\lower.7ex\hbox{E}\kern-.125emX}}
\begin{document}


\title{Light alarm - Using a react-native client in combination with a rust backend.}

\author{\IEEEauthorblockN{1\textsuperscript{st} Maksim Sandybekov}
\IEEEauthorblockA{\textit{computer science - autonomouse systems)} \\
\textit{HTWG Konstanz}\\
Konstaz, Germany \\
maksim.sandybekov@live.de}
\and
\IEEEauthorblockN{2\textsuperscript{nd} Benjamin Bäumler}
\IEEEauthorblockA{\textit{computer science - autonomouse systems} \\
\textit{HTWG Konstanz}\\
Konstanz, Germany \\
be391bae@htwg-konstanz.de}
}

\maketitle

\begin{abstract}

\end{abstract}

\begin{IEEEkeywords}
rust, react-native, redux, redux-saga, light, alarm, smart-light
\end{IEEEkeywords}

\section{Introduction}

\section{State of the art}


% Approach
\section{Proposed approach}


\section{architecture}

\subsection{interaction}


\section{client}


\section{server}
The main tasks of the server are to communicate with clients and to actually drive a 
led strip or any other kind of light source.

\subsection{Light Sources}

The server can support different kind of light sources, as long as it implements the \texttt{LedControls} trait.
Only one implementation of the trait can be used at the same time tho. But it is possible that a specific implementation
controls multiple hardware lights. At the current time there are only two implementations that can be used as light
sources:

\begin{itemize}
    \item LedStrip
    
    Controls a 4-pin led strip with a 12v pin and one pin for each color(red, green,blue). This cannot be driven 
    directly by the Raspberry Pi and therefore we need to use a extra curcuit board\cite{b0} for it. The circuit
    board allows us to controll each color seperatly by driving the gate pin of a MOSFET's respectivly. We will use 
    the pigpio Daemon\cite{b1} for this, because we need 3 PWM pins for this and currently available gpio libraries 
    for Rust only over up to 2 PWM pins.
    
    % TODO: add picture
    **picture of circuit board**

    \item MocLedStrip
    
    This is used only for testing. It allows us to verify the logic of the led controller without driving any
    GPIO's of the Raspberry Pi and also allows us to run the server on amd64 architecure for testing purposes.
\end{itemize}



\section{Results}


\section{Conclusion}


\section{Further work}



\begin{thebibliography}{00}
\bibitem{b0} Raspberry Pi \& RGB LED-Strips How to control a RGB LED-Strip with a Raspberry Pi By David Ordnung - https://dordnung.de/raspberrypi-ledstrip/
\bibitem{b1} pigpio Daemon - http://abyz.me.uk/rpi/pigpio/pigpiod.html

\end{thebibliography}
\vspace{12pt}

\end{document}
