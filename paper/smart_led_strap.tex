\documentclass[conference]{IEEEtran}
\IEEEoverridecommandlockouts
% The preceding line is only needed to identify funding in the first footnote. If that is unneeded, please comment it out.
\usepackage{cite}
\usepackage{amsmath,amssymb,amsfonts}
\usepackage{algorithmic}
\usepackage{graphicx}
\usepackage{textcomp}
\usepackage{xcolor}
\usepackage[utf8]{inputenc}
\def\BibTeX{{\rm B\kern-.05em{\sc i\kern-.025em b}\kern-.08em
    T\kern-.1667em\lower.7ex\hbox{E}\kern-.125emX}}
\begin{document}


\title{Wake alarm - Using a react-native client in combination with a rust backend.}

\author{\IEEEauthorblockN{1\textsuperscript{st} Maksim Sandybekov}
\IEEEauthorblockA{\textit{computer science - autonomouse systems)} \\
\textit{HTWG Konstanz}\\
Konstaz, Germany \\
maksim.sandybekov@live.de}
\and
\IEEEauthorblockN{2\textsuperscript{nd} Benjamin Bäumler}
\IEEEauthorblockA{\textit{computer science - autonomouse systems} \\
\textit{HTWG Konstanz}\\
Konstanz, Germany \\
be391bae@htwg-konstanz.de}
}


\maketitle

\begin{abstract}


\end{abstract}

\begin{IEEEkeywords}
rust, react-native, redux, redux-saga, light, alarm, smart-light
\end{IEEEkeywords}

% Motivation, why are we doing the likes
\section{Introduction}
The quality and duration of sleep affects the health and well being of individuals.
Additionally sleep plays a major role in consolidation of memory \cite{Rauchs2005} therefore it is essential for learning process.
Looking at studies on sleep deprevation and disorders it becomes clear that a poor sleep can cause an decrease in both
mental and physical performance. \cite{Mirghani2015a, Antunes2017a} Leading up to variouse physical and mental
diseases for example type-2 diabities, anxiety and increased depression. Especially students represent a group that 
is likely to suffer sever sleep deprevation. In an conducted study 46\% of 546 students rated their sleep as failry bad up to 
very bad. Furthermore 33\% of participants reported $\leq$ 7 h of sleep on study days with an average of 6.55 h. \cite{Norbury2019a}

The research surrounding the impact of light on the human circadian clock thereby revelead an relationship between illuminance and
alertness in human beings. \cite{DuffyJeanne2009a} In addition further research uncovered a link between exposure to more intense
light and the feeling of vitality during daytime and everyday situations. \cite{Smolders2014a}

These insights suggests a system that utilizes the effects of light on the human body to improve alertness and mental as well as
pyhsical performance during the day. Using these mechanisms within the context of circadian stimulation and sleep, different
fields of application become obviouse.

While technological development proceeds there are already attempts utilizing current innovations to harness prior introduced 
positive effects. One such attempt are wake lights that simulate the sunrise. A study investigating effects on the human body
concludes that simulation of the dawn significantly improves performance on attention- and motor-based tasks/skills during the day.
\cite{Gabel2015a}

This paper proposes an application implementing functionality to utilize previousely introduced advances in research 
concerning the effects of light on the human body. The application enables a user to obtain controll over a smart lamp
to regulate the illumination, color. An additional system for scheduling of illumination facilitates the simulation of sunrises.

\section{State of the art}
Currently there is a broad range of commercially available smart lamp devices and such. Each of these
devices uses a different technological stack and varies in their capabilities as well as their functionality.


% Approach
\section{Proposed approach}


\section{architecture}


\subsection{interaction}


\section{client}


\section{server}
The main tasks of the server are to communicate with clients and to actually drive a 
led strip or any other kind of light source.

\subsection{Light Sources}

The server can support different kind of light sources, as long as it implements the \texttt{LedControls} trait.
Only one implementation of the trait can be used at the same time tho. But it is possible that a specific implementation
controls multiple hardware lights. At the current time there are only two implementations that can be used as light
sources:

\begin{itemize}
    \item LedStrip
    
    Controls a 4-pin led strip with a 12v pin and one pin for each color(red, green,blue). This cannot be driven 
    directly by the Raspberry Pi and therefore we need to use a extra curcuit board\cite{b0} for it. The circuit
    board allows us to controll each color seperatly by driving the gate pin of a MOSFET's respectivly. We will use 
    the pigpio Daemon\cite{b1} for this, because we need 3 PWM pins for this and currently available gpio libraries 
    for Rust only over up to 2 PWM pins.
    
    % TODO: add picture
    **picture of circuit board**

    \item MocLedStrip
    
    This is used only for testing. It allows us to verify the logic of the led controller without driving any
    GPIO's of the Raspberry Pi and also allows us to run the server on amd64 architecure for testing purposes.
\end{itemize}



\section{Results}


\section{Conclusion}


\section{Further work}



% Add Links into bib file
% \begin{thebibliography}{00}
% \bibitem{b0} Raspberry Pi \& RGB LED-Strips How to control a RGB LED-Strip with a Raspberry Pi By David Ordnung - https://dordnung.de/raspberrypi-ledstrip/
% \bibitem{b1} pigpio Daemon - http://abyz.me.uk/rpi/pigpio/pigpiod.html
% \bibitem{a0} Smart LED Lamp with Bluetooth Speaker and Clock - https://store.earlham.edu/smart-led-lamp-bluetooth-speaker-and-clock
% \bibitem{a1} 
% \end{thebibliography}

\bibliographystyle{amsplain}
\bibliography{moco}


% \vspace{12pt}

\end{document}
